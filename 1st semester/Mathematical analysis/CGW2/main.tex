\documentclass{article}
\usepackage[utf8]{inputenc}
\usepackage[T1]{fontenc}
\usepackage[russian]{babel}
\usepackage{tikz}
\usepackage{graphicx}
\usepackage{titlesec}
\usepackage{amsfonts}
\usepackage{amsmath}
\usepackage{amssymb}
\usepackage[left=2cm,right=2cm,
    top=2cm,bottom=2cm,bindingoffset=0cm]{geometry}
\renewcommand{\thesection}{\arabic{section}}
\titleformat{\section}{\large\bfseries}{\thesection}{1em}{}
\title{Дифференцирование}
\author{Каренин Константин Витальевич}
\date{23.11.2023}
\begin{document}

\begin{titlepage}
    \centering
    \vspace*{0.5 cm}
    
    \textsc{\LARGE \textbf{Математический анализ}}
    \vspace{1.5cm}
    
    \rule{\linewidth}{0.2 mm} \\[0.4 cm]
    { \huge \bfseries Дифференцирование}
    \rule{\linewidth}{0.2 mm} \\[1.5 cm]
    
    \Large Выполнили: \\
    Каренин Константин \\
    Темиров Тимур \\
    Гонин Сергей \\
    
    \vspace{0.5cm}
    
    Группа: М3104
    
    \vspace{0.5cm}
    
    Преподаватель: Сарычев Павел
    
    \vspace{0.5cm}
    
    Университет ИТМО
    
    \vfill

    \includegraphics[height=70px]{logo.jpg}
    
    23.11.2023
    
\end{titlepage}

\setcounter{page}{2}

% task 1
\newpage
\section{$f(x)=\tg(x^3+x^2\sin\frac{2}{x})$}
    \subsection{Доопределение функции до непрерывности в точке $x_0=0$}
    $f(x_0) = \lim\limits_{x \to 0} f(x_0)$ - определение непрерывности функции в точке, Соответственно, чтобы функция была непрерывной  в точке $x_0$, необходимо, чтобы выполнялось определение непрерывности. Тогда, чтобы дополнить множество значений функции так, чтобы она была непрерывной, мы дополним его значением предела функции в $x_0$, тогда определение непрерывности фукции в точке будет выполняться \\ \\
    $\lim\limits_{x \to 0} \tg ( x^3 + x^2 \sin \frac{2}{x}) = \lim\limits_{x \to 0} \tg (0) = 0$ \\ \\
    $x^2 \sin \frac{2}{x}$ - стремится к 0 по теореме о произведении бесконечно малой на ограниченную функцию \\ \\
    Тогда $f(0) = 0$
    
    \subsection{Вычисление по определению $f'(0)$}
    $f'(0) = \lim\limits_{\Delta x \to 0} \frac{f(x + \Delta x) - f(x)}{\Delta x} = \lim\limits_{\Delta x \to 0} \frac{f(\Delta x)}{\Delta x} = \lim\limits_{\Delta x \to 0} \frac{\tg (\Delta x^3 + \Delta x^2 \sin \frac{2}{\Delta x})}{\Delta x} (\Delta x^3 + \Delta x^2 \sin \frac{2}{\Delta x}) =$ \\
    $= \lim\limits_{\Delta x \to 0} (\Delta x^2 + \Delta x \sin \frac{2}{\Delta x}) = 0$ \\ \\
    $\Delta x \sin \frac{2}{\Delta x}$ - стремится к 0 по теореме о произведении бесконечно малой на ограниченную функцию

    
% task 2
\newpage
\section{
\begin{equation*}
f(x) = 
\begin{cases}
    - \frac{(x+4)^2}{(x+3)^2}, x < -3 \\ \\ 
    1,56 \sqrt[3]{(x+2)^2} - 1,04 x - 2,08, x \geq -3 \\
\end{cases}
\end{equation*}
}
    \subsection{Поиск области определения функции}
    \begin{equation*} 
    \begin{cases}
        (x+3)^2 \neq 0 \\ 
        (x+2)^2 \geq 0 \\
    \end{cases}
    \begin{cases}
        x \neq -3 \; \text{из определения функции} \\ 
        x \in \mathbb{R} \\
    \end{cases}
    \Rightarrow
    x \in \mathbb{R}
    \end{equation*}
    
    \subsection{Исследование функции на чётность, нечётность и периодичность}
    Функуия чётна $\iff f(x) = f(-x)\; \forall x \in D(f)$ \\
    Функуия нечётна $\iff -f(x) = f(-x)\; \forall x \in D(f)$ \\
    \begin{equation*}
    f(-x) = 
    \begin{cases}
        - \frac{(x-4)^2}{(x-3)^2}, x > 3 \\
        1,56 \sqrt[3]{(x-2)^2} - 1,04 x - 2,08, x \leq -3 \\
    \end{cases}
    \end{equation*}
    \begin{equation*}
    -f(x) = 
    \begin{cases}
        \frac{(x+4)^2}{(x+3)^2}, x < -3 \\
        -1,56 \sqrt[3]{(x+2)^2} - 1,04 x - 2,08, x \geq -3 \\
    \end{cases}
    \end{equation*}
    $f(x) \neq f(-x) \Rightarrow f$ не чётная \\
    $-f(x) \neq f(-x) \Rightarrow f$ не нечётная \\

    \begin{equation*}
    f - \text{периодична} (T - \text{период}, T \in \mathbb{R}) \iff 
    \begin{cases}
        \forall x \in D(f) \iff x - T \in D(f) \iff x + T \in D(f) \\
        \forall x \in D(f) \; f(x) = f(x+T) = f(x-T) \\
    \end{cases}
    \end{equation*}
    $x \in (- \infty;-4) \; f - \text{монотонна} \Rightarrow \forall x_1, x_2 \in D(f) \; | \; x_1 < x_2 \Rightarrow f(x_1) \neq f(x_2) \iff f(x_1) - f(x_2) \neq 0$ \\
    $x < x + T \Rightarrow f(x) - f(x+T) \neq 0 \Rightarrow f(x) \neq f(x+T) \Rightarrow$ функция не периодичная
    
    \subsection{Поиск точек пересечения графика функции с координатными осями}
    $f(0) = 1,56 \sqrt[3]{(0+2)^2} - 1,04 \cdot 0 - 2,08 = \sqrt[3]{4} - 2,08$ - пересечение с осью ординат \\
    $f(x) = 0$ - пересечение с осью ординат, соответсвенно точки пересечения: \\
    \begin{equation*}
    \left[ 
      \begin{gathered} 
        - \frac{(x+4)^2}{(x+3)^2} = 0 \\ 
        1,56 \sqrt[3]{(x+2)^2} - 1,04 x - 2,08 = 0 \\
      \end{gathered} 
    \right.
    \left[ 
      \begin{gathered} 
        x = -4 \\ 
        x = \frac{11}{8} \\
      \end{gathered} 
    \right.
    \end{equation*}    

    \subsection{Исследование функции на непрерывность: нахождение точек разрыва и типа разрыва в них}
    $f(-3) = 1,56 \sqrt[3]{(-3+2)^2} - 1,04 \cdot (-3) - 2,08 = 1,56 + 3 \cdot 1,04 - 2,08 = 7,6$ \\
    $\lim\limits_{x \to -3 - 0} f(x) = \lim\limits_{x \to -3 - 0} - \frac{(x+4)^2}{(x+3)^2} = \lim\limits_{x \to -3 - 0} - \frac{(-3-0+4)^2}{(-3+3-0)^2} = - \infty$ \\
    неустранимый разрыв второго рода

    \subsection{Поиск асимптот графика функции}
    Формула наклонной асимтоты: \\
    $y = kx + b$, где: \\
    $k = \lim\limits_{x \to \infty} \frac{f(x)}{x}$ \\
    $b = \lim\limits_{x \to \infty} (f(x) -k(x))$ \\ \\
    Для $x \to - \infty$: \\ 
    $k = \lim\limits_{x \to - \infty} \frac{- \frac{(x+4)^2}{(x+3)^2}}{x} = \frac{-1}{- \infty} = 0$ \\
    $b = \lim\limits_{x \to - \infty} - \frac{(x+4)^2}{(x+3)^2} - 0 = -1$ \\
    Горизонтальная асимтота $y = -1$ \\ \\
    Для $x \to + \infty$: \\
    $k = \lim\limits_{x \to + \infty} \frac{1,56 \sqrt[3]{(x+2)^2} - 1,04 x - 2,08}{x} = -1$ \\
    $b = \lim\limits_{x \to + \infty} = 1,56 \sqrt[3]{(x+2)^2} - 1,04 x - 2,08 - 1 = - \infty$ \\
    Горизонтальная асимтота не существует
    
    \subsection{Поиск промежутков возрастания, убывания и экстремумов функции}
    \begin{equation*}
    f'(x) = 
    \begin{cases}
        \frac{2(x+4)(x+3)^2 - 2(x+3)(x+4)^2}{(x+3)^4}, x < -3 \\ \\ 
        1,56 \cdot \frac{2}{3}(x+2)^{-\frac{1}{3}} - 1,04, x \geq -3 \\
    \end{cases}
    =
    \begin{cases}
        \frac{2(x+4)(x+3)(x+3-x-4)}{(x+3)^4}, x < -3 \\ \\ 
        1,56 \cdot \frac{2}{3}(x+2)^{-\frac{1}{3}} - 1,04, x \geq -3 \\
    \end{cases}
    \begin{cases}
        \frac{2(x+4)(x+3)}{(x+3)^3}, x < -3 \\
        1,56 \cdot \frac{2}{3}(x+2)^{-\frac{1}{3}} - 1,04, x \geq -3 \\
    \end{cases}
    \end{equation*}
    Функция возрастает на определённом отрезкке/интервале, если производная больше 0 и убывает, если производная меньше 0
    \begin{equation*}
    \begin{cases}    
        \begin{cases}
            \frac{2(x+4)(x+3}{(x+3)^3} > 0 \\
            x < -3 \\
        \end{cases} \\
        \begin{cases}
            1,56 \cdot \frac{2}{3}(x+2)^{-\frac{1}{3}} - 1,04 > 0 \\
            x \geq -3 \\
        \end{cases}
    \end{cases}
    \end{equation*}
    \begin{equation*}
    \begin{cases}
        \begin{cases}
            (x+2)^\frac{1}{3} > 0 \\
            (x+2)^\frac{1}{3} < 1 \\            
        \end{cases} \\
        \begin{cases}
            \frac{2(x+4)(x+3)}{(x+3)^3} > 0 \\
            x < -3 \\          
        \end{cases}
    \end{cases}
    \begin{cases}
        \begin{cases}
            x > -2 \\
            x < -1 \\            
        \end{cases} \\
        x \in (- \infty; -4) \\
      \end{cases} 
    \end{equation*}
    Точка -2 это точка подозрительная на экстремум, поскольку в ней не существует производной \\
    Точки максимума: $x=-4; x=-1$ \\
    Точки минимума: $x=-2$ \\
    Возрастает на $x \in (- \infty; -4)$ и $(-2;-1)$ \\
    Убывает на $x \in (-4;-3)$ и $(-3;-2)$ и $(-1; + \infty)$
    
    \subsection{Поиск промежутков выпуклости и точек перегиба функции}
    \begin{equation*}
    f'(x) = 
    \begin{cases}
        - \frac{2(x+8)}{(x+3)^3}, x < -3 \\
        1,56 \cdot \frac{2}{3}(x+2)^{-\frac{1}{3}} - 1,04, x \geq -3 \\
    \end{cases}
    \end{equation*}
    \begin{equation*}
    f''(x) = 
    \begin{cases}
        - \frac{2(x+3)^3 - (2x+8) \cdot 3(x+3)^2}{(x+3)^6}, x < -3 \\
        1,04 \cdot -\frac{1}{3}(x+2)^{-\frac{4}{3}}, x \geq -3 \\
    \end{cases}
    = 
    \begin{cases}
        - \frac{(x+3)^2(2x+6-6x-24)}{(x+3)^6}, x < -3 \\
        - \frac{1,04}{3 \sqrt[3]{(x+2)^4}}, x \geq -3 \\
    \end{cases}
    = 
    \begin{cases}
        - \frac{2(2x+9)}{(x+3)^4}, x < -3 \\
        - \frac{1,04}{3 \sqrt[3]{(x+2)^4}}, x \geq -3 \\
    \end{cases}
    \end{equation*}
    \begin{equation*}
    \begin{cases}
        - \frac{2(2x+9)}{(x+3)^4}, x < -3 \\
        x < -3 \\
    \end{cases}
    \end{equation*}
    \begin{equation*}
    \begin{cases}
        - \frac{1,04}{3 \sqrt[3]{(x+2)^4}} \\ 
        x \geq -3 \\
    \end{cases}
    \Rightarrow
    \begin{cases}
        \left[ 
          \begin{gathered} 
            \begin{cases}
                x + 2 < 0 \\
                \sqrt[3]{x+2} > 0 \\            
            \end{cases} \\
            \begin{cases}
                x + 2 > 0 \\
                \sqrt[3]{x+2} < 0 \\            
            \end{cases}
          \end{gathered} 
        \right. \\
        x \geq -3 \\
    \end{cases}
    \Rightarrow
    \varnothing
    \end{equation*}
    Выпуклая на $(-4,5; -3)$ и $(-3; -2)$ и $(-2; + \infty)$ \\
    Вогнутая на $(- \infty; -4,5)$ \\
    Точки перегиба: $x=-4,5$
    


% task 3
\newpage   
\section{$f(x) = (2x + 3) e^{4x^2+4x-3}, x_0 = \frac{1}{2}, k = 2n + 1$}
$\sqsupset f(x)=a(x)g(x), \text{где:}$ \\ \\
$a(x) = 2x+3 $ \\ \\
$g(x) = e^{(2x+3)(2x-1)} $ \\ \\
Рассмотрим производные функции f(x) для некоторых порядков: \\ \\
$f'(x_0) = a'(x_0)g(x_0)+g'(x_0)a(x_0) $ \\ \\
$f''(x_0) = (a'(x_0)g(x_0)+g'(x_0)a(x_0))' = a''(x_0)g(x_0)+2a'(x_0)g'(x_0)+a(x_0)g''(x_0) $ \\ \\
$f'''(x_0) = a'''(x_0)g(x_0)+a''(x_0)g'(x_0)+2a''(x_0)g'(x_0)+2a'(x_0)g''(x_0)+a'(x_0)g''(x_0)+a(x_0)g'''(x_0) = a'''(x_0)g(x_0)+3a''(x_0)g'(x_0)+3a'(x_0)g''(x_0)+a(x_0)g'''(x_0)$ \\ \\
Теперь нетрудно заметить биномиальную запись производной 
p-го порядка: \\ \\
$f^{(p)}(x_0) = \sum_{i=0}^{p} C_{p}^{i}a^{(p-i)}(x_0)g^{(i)}(x_0)$ \\ \\
Тогда мы можем вывести формулу Тейлора для общего случая: \\ \\
$f(x) = f(x_0) + f'(x_0)(x-x_0) + \sum_{j=2}^{k}(\frac{f^{(j)}(x_0)}{j!}(x-x_0)^j)+o((x-x_0)^{k})$ \\ \\
Тогда подставим численные значения в формулу и получим (о Боже!): \\ \\
$f(x)=4+34(x-0.5)+\sum_{j=2}^{2n+1}(\frac{\sum_{i=0}^{j} C_{j}^{i}a^{(j-i)}(0.5)g^{(i)}(0.5)}{j!}(x-0.5)^j) + o((x-0.5)^{2n+1}), \text{ где $n$ - определённая точность значения}$

    
% evaluation paper
\newpage
\[
\renewcommand{\arraystretch}{2}
\begin{tabular}{| c | c |}
 \hline
    \hugeУчастник & \hugeВклад в \% \\
 \hline
    \hugeКаренин Константин & \huge33.(3) \\
 \hline
    \hugeГонин Сергей & \huge33.(3) \\
 \hline
    \hugeТемиров Тимур & \huge33.(3) \\
 \hline
\end{tabular}
\]
\end{document}